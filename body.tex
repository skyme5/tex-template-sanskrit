\chapter{अथ प्रथमो डध्यायः}

\begin{center}
	अथः प्रथमः अध्यायः | now the first chapter
\end{center}

\begin{verse}
धृतराष्ट्र उवाच

धर्मक्षेत्रे कुरुक्षेत्रे समवेता युयुत्सवः |
मामकाः पाण्डवाश्र्चैव किमकुर्वत सज्जय |१|
\end{verse}
"Dhrtarastra said: O Sanjaya, after my sons and the sons of Pandu assembled in the place of pilgrimage at Kuruksetra, desiring to fight, what did they do?"

\begin{center}

सर्वधर्मान् परित्यज्य मामेकं शरणं व्र्ज\\
अहं त्वां सर्वपापेभ्यो मोक्षयिष्यामि मा शुचः

\end{center}

नई दिल्ली, भारत की राजधानी है। कुल ४२.७ वर्ग किमी क्षेत्रफल के साथ, नई दिल्ली दिल्ली महानगर
के भीतर आता है और यहाँ पर भारत सरकार और दिल्ली सरकार के सभी प्रशासनिक भवन स्थित हैं।

नई दिल्ली, भारत की राजधानी है। कुल ४२.७ वर्ग किमी क्षेत्रफल के साथ, नई दिल्ली दिल्ली महानगर
के भीतर आता है और यहाँ पर भारत सरकार और दिल्ली सरकार के सभी प्रशासनिक भवन स्थित हैं।

\chapter{श्लोकः}

कराग्रे वसते लक्ष्मीः, कर मध्ये सरस्वती |
करमूले तू गोविन्दः, प्रभाते कर दर्शनम्  ||

समुद्रवसने ! देवि ! पर्वत-स्तन-मण्डले |
विष्णु-पत्नि ! नमस्तुभ्यम्, पाद-स्पर्षँ क्शमस्वमे |


कराग्रे वसते लक्ष्मीः कर मूले सरस्वती